
\documentclass{article}
\usepackage{amsmath, amssymb}

\title{Stage 11: Warping the Manifold (Well Parser)}
\author{NGeodesic Project}
\date{\today}

\begin{document}
\maketitle

\section*{Stepwise Plan for Stage 11}

This stage formalizes the ``well'' perspective by turning the parser into an explicit
energy minimization problem. The goal is to warp the manifold into a deep basin that
suppresses hallucinations and stabilizes true solutions.

\subsection*{Step 1. Formalize the Energy Landscape}
Define a per-primitive potential
\begin{equation}
U_p = -\left( w_\perp z_{\perp,p} + w_{\text{raw}} z_{\text{raw},p} - w_{\text{cm}} z_{\text{cm},p} \right).
\end{equation}
The margin
\begin{equation}
\Delta = U_{\text{false}} - U_{\text{true}}
\end{equation}
is the measure of well depth.

\subsection*{Step 2. Improve Orthogonalization}
Replace mean-only subtraction with low-rank projection. Compute top-$r$ principal
components across primitives, then project onto the orthogonal complement. This enlarges
$\Delta$ by removing more shared variance.

\subsection*{Step 3. Strengthen Calibration}
Enhance the permutation null:
\begin{itemize}
    \item Use block circular shifts to preserve autocorrelation.
    \item Optionally apply multitaper averaging for variance reduction.
\end{itemize}

\subsection*{Step 4. Sequential Residual Refinement}
After selecting a primitive, subtract its prototype contribution from all channels and
recompute $z$-scores. True picks reduce residual energy, deepening the well; false picks do not.

\subsection*{Step 5. Lateral Inhibition}
Introduce penalties when two primitives activate in overlapping windows. Implement as a
Gaussian kernel $\kappa(t_i, t_j)$ in the energy function to suppress distractor basins.

\subsection*{Step 6. Controlled Descent}
Introduce a temperature parameter $T$:
\begin{equation}
\pi(p) \propto \exp(-U_p / T).
\end{equation}
Start with high $T$ (exploration) and anneal to low $T$ (deep well, stable convergence).

\subsection*{Step 7. Evaluation Metrics}
Track margin distributions and performance metrics as each component is added:
\begin{itemize}
    \item Hallucination rate should decrease steadily.
    \item Recall should remain stable ($\pm 2\%$).
    \item $F_1$ and Jaccard scores should climb.
    \item Exact grid accuracy should rise meaningfully.
\end{itemize}

\section*{Breaking Point}
Stage 11 represents the \textbf{breaking point} of the project: the transition from
heuristic parsing to an explicit warped manifold energy framework. This is where
hallucinations are eliminated by construction.

\end{document}
